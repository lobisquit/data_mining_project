%%%%%%%%%%%%%%%%%%%%%%%%%%%%%%%%%%%%%%%%%
% Arsclassica Article
% LaTeX Template
% Version 1.1 (10/6/14)
%
% This template has been downloaded from:
% http://www.LaTeXTemplates.com
%
% Original author:
% Lorenzo Pantieri (http://www.lorenzopantieri.net) with extensive modifications by:
% Vel (vel@latextemplates.com)
%
% License:
% CC BY-NC-SA 3.0 (http://creativecommons.org/licenses/by-nc-sa/3.0/)
%
%%%%%%%%%%%%%%%%%%%%%%%%%%%%%%%%%%%%%%%%%

%----------------------------------------------------------------------------------------
%	PACKAGES AND OTHER DOCUMENT CONFIGURATIONS
%----------------------------------------------------------------------------------------

\documentclass[
12pt, % Main document font size
a4paper, % Paper type, use 'letterpaper' for US Letter paper
oneside, % One page layout (no page indentation)
%twoside, % Two page layout (page indentation for binding and different headers)
headinclude,footinclude, % Extra spacing for the header and footer
BCOR5mm, % Binding correction
]{scrartcl}

\input{structure.tex} % Include the structure.tex file which specified the document structure and layout

\hyphenation{Fortran hy-phen-ation} % Specify custom hyphenation points in words with dashes where you would like hyphenation to occur, or alternatively, don't put any dashes in a word to stop hyphenation altogether

%----------------------------------------------------------------------------------------
%	TITLE AND AUTHOR(S)
%----------------------------------------------------------------------------------------

\title{\normalfont\spacedallcaps{Generazione delle categorie di wikipedia attraverso il clustering}} % The article title

\author{\spacedlowsmallcaps{Cazzaro Dalla Cia Lovisotto Vianello}}

\date{} % An optional date to appear under the author(s)

%----------------------------------------------------------------------------------------

\begin{document}

%----------------------------------------------------------------------------------------
%	HEADERS
%----------------------------------------------------------------------------------------

\renewcommand{\sectionmark}[1]{\markright{\spacedlowsmallcaps{#1}}} % The header for all pages (oneside) or for even pages (twoside)
%\renewcommand{\subsectionmark}[1]{\markright{\thesubsection~#1}} % Uncomment when using the twoside option - this modifies the header on odd pages
\lehead{\mbox{\llap{\small\thepage\kern1em\color{halfgray} \vline}\color{halfgray}\hspace{0.5em}\rightmark\hfil}} % The header style

\pagestyle{scrheadings} % Enable the headers specified in this block

%----------------------------------------------------------------------------------------
%	TABLE OF CONTENTS & LISTS OF FIGURES AND TABLES
%----------------------------------------------------------------------------------------

\maketitle % Print the title/author/date block





\newpage

%----------------------------------------------------------------------------------------
%	INTRODUCTION
%----------------------------------------------------------------------------------------

\section{Introduzione}

- descrizione del problema in generale e del constesto

- delineazione degli obiettivi:

	- gli articoli di wikipedia sono clusterizzabili?

	- rapporto tra cluster e le categorie?

	- cosa succede con varie tecniche di clustering?


\section{Dataset e Analisi Preliminare}

	- descrizione del dataset
	
	- pulizia degli articoli (senza categorie, disambigua)
	
	- sort e distribuzione delle categorie


\section{Rappresentazione del Dataset}

	- eliminazione delle stopword e lemmatization (sia per wiki vectors che per bag of words)

	\subsection{Vettoriale}

		- allenamento word2vec con i suoi vari parametri (dare una motivazione per la dimensione 100)

		- trasformazione delle parole in vettori-articolo

	\subsection{Bag of Words}

		- la trasformazione in bag of words



\section{Clustering}

	\subsection{Tecniche di Clustering}

		\subsubsection{Hopkins Statistic}

			Riportiamo lo score che ci dice che il nostro dataset è ben clusterizzabile

		\subsubsection{Kmeans}

			Kmeans e il suo score con un bel grafico

		\subsubsection{Altri metodi}

			Abbiamo provato anche il clustering gerarchico e il Gaussian Mixture Model
			ma non abbiamo abbastanza potenza di calcolo

		\subsubsection{Latent Dirichlet Allocation}

			Vediamo cosa viene fuori e un bel grafico


	\subsection{Valutazioni del Clustering}

		\subsubsection{Simple Silhouette}

			Utilizzo della versione semplificata di Silhouette con i centroidi
			Rimozione dei cluster con un solo articolo dal punteggio
			Magari buttiamoci un peso a sta metrica

		\subsubsection{Normalized Mutual Information}

			Il problema di individuare una funzione obiettivo che utilizzi le categorie,
			le quali non formano una partizione in quanto overlapping
			Pulizia delle categorie con con idf
			Modifica dell'algoritmo



\section{Risultati}

	\subsection{Numero di cluster}

		Il K selezionato dalle due tecniche Kmeans e LDA
		Speriamo sia simile!

	\subsection{Validazione con Simple Silhouette}

		L'andamento sempre crescente della Silhouette
		Un bel grafico lineare

	\subsection{Confronto tra i Cluster ottenuti con Normalize Mutual Information}

		Confronto tramite NMI delle due tecniche Kmeans e LDA


\section{Conclusioni}

	Le conclusioni generali dalle analisi effettuate

	Proposte di punti da approfondire in studi futuri




%----------------------------------------------------------------------------------------
%	BIBLIOGRAPHY
%----------------------------------------------------------------------------------------

\renewcommand{\refname}{\spacedlowsmallcaps{References}} % For modifying the bibliography heading

\bibliographystyle{unsrt}

\bibliography{sample.bib} % The file containing the bibliography

%----------------------------------------------------------------------------------------

\end{document}
